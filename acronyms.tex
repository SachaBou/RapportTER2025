\chapter*{Acronyms}

\textbf{GCR: } Galactic Cosmic Ray\\
\textbf{CEO: } Chief Executive Officer\\
\textbf{CSDA: } CSDA range: a very close approximation to the average path length traveled by a charged particle as it slows down to rest, calculated in the continuous-slowing-down approximation. In this approximation, the rate of energy loss at every point along the track is assumed to be equal to the total stopping power. Energy-loss fluctuations are neglected. The CSDA range is obtained by integrating the reciprocal of the total stopping power with respect to energy.\\
\textbf{HZETRN: } The HZETRN (High charge (Z) and Energy TRaNsport) code is a deterministic transport model designed by NASA specifically for simulating space radiation transport\\
\textbf{HZE: } The abbreviation comes from high (H), atomic number (Z), and energy (E)\\
\textbf{IAEA: } International Atomic Energy Agency\\
\textbf{ICRP: } International Commission on Radiological Protection\\
\textbf{LEO: } Low Earth Orbit, between 160km and 2000km\\
\textbf{LMO: } Low Mars Orbit, similar to LEO\\
\textbf{NASA: } National Aeronautics and Space Administration\\
\textbf{NIST: } National Institute of Standards and Technology\\
\textbf{NTR: } Nuclear Thermal Rocket\\
\textbf{SEP: } Solar Energetic Particles\\
\textbf{TIPS: } Transjugular intrahepatic portosystemic shunt, this procedure is done by an interventional radiologist under x-ray guidance


\newpage
